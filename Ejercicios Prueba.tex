\documentclass[spanish,12pt]{article}

\usepackage{enumitem}
\usepackage{tikz}
\usepackage{pgfplots}
\usepackage{amssymb}
\usepackage{booktabs}
\usepackage{enumerate}
\usepackage{multirow}
\usepackage{amsmath}
\usepackage{graphicx}
\usepackage[margin=1in]{geometry}
\usepackage{indentfirst}
\usepackage[spanish]{babel}
\usepackage[utf8]{inputenc}
\usepackage{fancyhdr}
\usepackage{tikz-feynman}
\usepackage{physics}
\usepackage{color}
\renewcommand{\baselinestretch}{1.5}

\newcommand{\dydx}{\frac{dy}{dx}}
\newcommand{\ddx}{\frac{dy}{dx}}
\newcommand{\R}{\mathbb{R}}
\newcommand{\N}{\mathbb{R}}
\newcommand*\Eval[3]{\left.#1\right\rvert_{#2}^{#3}}

\setlength{\parskip}{0.4cm}

\pagestyle{fancy}
\fancyhead{}
\fancyfoot{}
\usepackage{titlesec}
\titleformat{\section}
  {\normalfont\Large\bfseries}{\thesection}{1em}{}[{\titlerule[0.8pt]}]
\usepackage{hyperref}
\hypersetup{
    colorlinks=true,
    linkcolor=blue,
    filecolor=magenta,      
    urlcolor=cyan,
    pdftitle={Solucionario P3},
}
\fancyhead[l]{\fontsize{8}{12}\slshape\MakeUppercase{Ejercicios: Demostración}}
\fancyhead[R]{\slshape{J. Pulido}}
\fancyfoot[c]{\thepage}
\pgfplotsset{compat=1.17}
\begin{document}
	\begin{titlepage}
	\begin{center}
	\hspace{0pt}
	\vfill
	{\Large\textbf{{Ejercicios: Demostración}}}
	
	\thispagestyle{empty}
	\vfill
	\end{center}
	\end{titlepage}
\newpage
\section{Ejercicios}
\begin{enumerate}[1)]
    \item Demuestre por contradicción que existen infinitos primos.
    \item Demuestre que la raíz de cualquier primo es irracional.
    \item Demuestre por prueba directa la identidad de Pascal.
    \item Demuestre por inducción el teorema del binomio para exponentes naturales (\textit{tip.} utilice la identidad de Pascal).
    \item Demuestre por contradicción la validez del método inductivo de prueba (\textit{tip.} Puede desarrollarlo con palabras o teoría de conjuntos)
    \item Demuestre las siguientes propiedades de potencias y logaritmos
    \begin{enumerate}
        \item $a^0=1\quad\forall\;a\neq0$
        \item $p^{\log_pa}=a$
        \item $\log_{p}a+\log_{p}b=\log_{p}(ab)$
        \item $\log_{p}a-\log_{p}b=\log_{p}(\frac{a}{b})$
        \item $\log_p1=0$
    \end{enumerate}
    \item Demuestre por inducción que $x^n-1=(x-1)(x^{n-1}+x^{n-2}+x^{n-3}+...x^2+x+1)$
    \item Dado lo anterior demuestre que el antecesor de cualquier potencia natural de un natural $a$ es divisible por el antecesor de $a$.
    
\end{enumerate}
\end{document}