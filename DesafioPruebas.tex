\documentclass[spanish,12pt]{article}

\usepackage{enumitem}
\usepackage{tikz}
\usepackage{pgfplots}
\usepackage{amssymb}
\usepackage{booktabs}
\usepackage{enumerate}
\usepackage{multirow}
\usepackage{amsmath}
\usepackage{graphicx}
\usepackage[margin=1in]{geometry}
\usepackage{indentfirst}
\usepackage[spanish]{babel}
\usepackage[utf8]{inputenc}
\usepackage{fancyhdr}
\usepackage{tikz-feynman}
\usepackage{physics}
\usepackage{color}
\renewcommand{\baselinestretch}{1.5}

\newcommand{\dydx}{\frac{dy}{dx}}
\newcommand{\ddx}{\frac{dy}{dx}}
\newcommand{\R}{\mathbb{R}}
\newcommand{\N}{\mathbb{R}}
\newcommand*\Eval[3]{\left.#1\right\rvert_{#2}^{#3}}

\setlength{\parskip}{0.4cm}

\pagestyle{fancy}
\fancyhead{}
\fancyfoot{}
\usepackage{titlesec}
\titleformat{\section}
  {\normalfont\Large\bfseries}{\thesection}{1em}{}[{\titlerule[0.8pt]}]
\usepackage{hyperref}
\hypersetup{
    colorlinks=true,
    linkcolor=blue,
    filecolor=magenta,      
    urlcolor=cyan,
    pdftitle={Solucionario P3},
}
\fancyhead[l]{\fontsize{8}{12}\slshape\MakeUppercase{Solucionario ejercicio P3}}
\fancyhead[R]{\slshape{S.Morales y J. Pulido}}
\fancyfoot[c]{\thepage}
\pgfplotsset{compat=1.17}
\begin{document}
	\begin{titlepage}
	\begin{center}
	\hspace{0pt}
	\vfill
	{\Large\textbf{{Desafío}}}
	
	\thispagestyle{empty}
	\vfill
	\end{center}
	\end{titlepage}
\newpage
\section{Desafío}

El teorema fundamental del álgebra establece que una ecuación degrado $n$ tiene exactamente $n$ raíces (complejas). Notese que el grupo de complejos engloba a los números reales, por lo tanto una ecuación con coeficientes reales de grado $n$ tiene también $n$ soluciones en los números complejos.

En este desafío, deberá probar que todo polinomio con coeficientes reales puede ser expresado en función de factores lineales y cuadráticos. En otras palabras, cualquier ecuación de grado 3 o superior se puede factorizar en términos de menor grado.

Para comenzar,  pruebe las siguientes identidades en los números complejos:

Sea $z$ un numero complejo y $\Bar{z}$ su conjugado:
\begin{align*}
    \Bar{z} \cdot \Bar{x} &= \overline{z\cdot x}\\
    \overline{z}+\overline{x}&=\overline{z+x}
\end{align*}

Ahora, considere la ecuación
\begin{align*}
    f(x)=a_1x^n+a_2x^{n-1}+\dots+a_{n-1}x+a_n
\end{align*}

Con $\{a_n\}_{n \in \mathbb{N}}\in \mathbb{R}$

Cuantas soluciones tiene la ecuación $f(x)=0$?

A partir de esto, escriba la factorizacion, en los números complejos, de $f(x)$ (Puede expresar los coeficientes como $k_n$).

Pruebe que $\overline{f(x)}=f(\overline{x})$

A partir de esto, explique porque todo polinomio de coeficientes reales se puede factorizar en una combinación de términos lineales y cuadráticos.
\newpage
\section{Solución}

La prueba de las identidades es trivial y no se explicara.

Que $f(x)=0$ significa que el valor de $x$ valoriza a la función compleja $f$ en $0$, es decir las raíces de la ecuación. Por lo tanto, según el teorema fundamental del álgebra $f(x)$ tendrá $n$ raíces.

Entonces, $f(x)$ se puede factorizar como:
\begin{align*}
    f(x)=(x-k_1)(x-k_2)\cdots(x-k_n)
\end{align*}

Utilizando las identidades probadas al comienzo de esta seccion

\begin{align*}
    \overline{f(x)}&=\overline{a_1x^n+a_2x^{n-1}+\dots+a_{n-1}x+a_n}\\
    \overline{f(x)}&= \overline{a_1x^n}+\overline{a_2x^{n-1}}+\cdots+\overline{a_{n-1}x}+\overline{a_n}
\end{align*}

Evaluando solo un termino, debido a que es suficiente
\begin{align*}
    \overline{a_1x^n}&=\overline{a_1}\overline{x^n}\\
    \overline{a_1}\overline{x^n}&=a_1\overline{x}^n\\
    \overline{f(x)}&=f(\overline{x})
\end{align*}
\hfill $\square$

Ahora, si $k_s \in \mathbb{R}$, entonces tenemos un termino lineal en nuestra factorizacion, si $k_s$ es un complejo con su parte imaginaria distinta de cero, sabemos que su conjugado tambien sera un factor, por lo tanto 2 factores de $f(x)$ serán 

\begin{align*}
    (x-k_s)(x-\overline{k_s})=(x^2-x(k_s+\overline{k_s})+k_s\overline{k_s})
\end{align*}

Dado que $k_S=a+bi$ con $a,b \in \mathbb{R}$

\begin{align*}
    k_s+\overline{k_s}&=2a\\
    k_s\overline{k_S}&=(a+bi)(a-bi)=(a^2+b^2)
\end{align*}

Entonces

\begin{align*}
     (x-k_s)(x-\overline{k_s})= (x^2-2ax+a^2+b^2)
\end{align*}

Es una expresión cuadrática con coeficientes reales

\hfill $\square$
\end{document}