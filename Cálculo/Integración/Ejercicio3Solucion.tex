\documentclass[spanish,12pt]{article}

\usepackage{enumitem}
\usepackage{tikz}
\usepackage{pgfplots}
\usepackage{amssymb}
\usepackage{booktabs}
\usepackage{enumerate}
\usepackage{multirow}
\usepackage{amsmath}
\usepackage{graphicx}
\usepackage[margin=1in]{geometry}
\usepackage{indentfirst}
\usepackage[spanish]{babel}
\usepackage[utf8]{inputenc}
\usepackage{fancyhdr}
\usepackage{tikz-feynman}
\usepackage{physics}
\usepackage{color}
\renewcommand{\baselinestretch}{1.5}
\newcommand{\dydx}{\frac{dy}{dx}}
\newcommand{\R}{\mathbb{R}}
\setlength{\parskip}{0.4cm}

\pagestyle{fancy}
\fancyhead{}
\fancyfoot{}
\usepackage{titlesec}
\titleformat{\section}
  {\normalfont\Large\bfseries}{\thesection}{1em}{}[{\titlerule[0.8pt]}]
\usepackage{hyperref}
\hypersetup{
    colorlinks=true,
    linkcolor=blue,
    filecolor=magenta,      
    urlcolor=cyan,
    pdftitle={IB Cálculo},
}
\fancyhead[l]{\fontsize{8}{12}\slshape\MakeUppercase{IB Cálculo}}
\fancyhead[R]{\slshape{S.Morales y J. Pulido}}
\fancyfoot[c]{\thepage}
\pgfplotsset{compat=1.17}
\begin{document}
	\begin{titlepage}
	\begin{center}
	\hspace{0pt}
	\vfill
	{\Large\textbf{{IB Cálculo}}}
	
	\medskip
	Solucion Ejercicio 3
	
	\medskip
    Samuel Morales y Julio Pulido
	
	\thispagestyle{empty}
	\vfill
	\end{center}
	\end{titlepage}
\newpage
\tableofcontents
\newpage
\section{Sección a}

\begin{align*}
    \int_a^b f(x)dx= \int_a^b f(a+b-x) dx
\end{align*}

En nuestra integral sustituimos $x=a+b-u$, $dx=-du$, notese que nuestros nuevos limites de integración son $b$ y $a$ (calculando el valor de $u$ en la sustitución para $x=a$ t $x=b$)
\begin{align*}
    \int_a^b f(x)dx&= \int_b^a -f(a+b-u) du\\
    -\int_b^a f(a+b-u) du&=\int_a^b f(a+b-u) du
\end{align*}
\hfill $\square$

Como $u$ es una variable de integración, esta puede ser reemplazada por $x$, terminando el ejercicio (si no se siente convencido haga la sustitución $x=u$)

\section{Seccion b}
\begin{align*}
    I&=\int_0^{\frac{\pi}{2}}\frac{\cos{x}}{\sin{x+\cos{x}}}dx=\int_0^{\frac{\pi}{2}}\frac{\sin{x}}{\cos{x}+\sin{x}}dx
\end{align*}

Este ultimo paso fue aplicar la propiedad derivada en la seccion anterior, pues 

$$\cos{\left(\frac{pi}{2}-x\right)}=\sin{x}$$

Entonces:
\begin{align*}
    2I&=\int_0^{\frac{\pi}{2}}\frac{\cos{x}}{\sin{x+\cos{x}}}dx+\int_0^{\frac{\pi}{2}}\frac{\sin{x}}{\cos{x}+\sin{x}}dx\\
    2I&=\int_0^{\frac{\pi}{2}}\frac{\cos(x)+\sin(x)}{\cos(x)+\sin(x)}dx\\
    2I&=\int_0^{\frac{\pi}{2}} dx=\frac{\pi}{2}\\
    I&=\frac{\pi}{4}
\end{align*}
\hfill $\square$

\section{Seccion c}
\begin{align*}
    \int_0^{\frac{\pi}{2}}\frac{1}{1+\tan(x)}dx&=\int_0^{\frac{\pi}{2}}\frac{1}{1+\frac{\sin(x)}{\cos(x)}}dx\\
    \int_0^{\frac{\pi}{2}}\frac{1}{1+\frac{\sin(x)}{\cos(x)}}dx&=\int_0^{\frac{\pi}{2}}\frac{1}{\frac{\cos{(x)}+\sin(x)}{\cos(x)}}dx\\
    \int_0^{\frac{\pi}{2}}\frac{1}{\frac{\cos{(x)}+\sin(x)}{\cos(x)}}dx&=\int_0^{\frac{\pi}{2}}\frac{\cos{(x)}}{\cos{(x)}+\sin(x)}dx= \frac{\pi}{4}
\end{align*}
\hfill $\square$

\section{Seccion d}
\begin{align*}
  I=\int_0^{\frac{\pi}{2}}\frac{1}{1+\tan^{2021}(x)}dx&=\int_0^{\frac{\pi}{2}}\frac{1}{1+\frac{\sin^{2021}(x)}{\cos^{2021}(x)}}dx\\
    \int_0^{\frac{\pi}{2}}\frac{1}{1+\frac{\sin^{2021}(x)}{\cos^{2021}(x)}}dx&=\int_0^{\frac{\pi}{2}}\frac{1}{\frac{\cos^{2021}{(x)}+\sin^{2021}(x)}{\cos^{2021}(x)}}dx\\
    \int_0^{\frac{\pi}{2}}\frac{1}{\frac{\cos^{2021}{(x)}+\sin^{2021}(x)}{\cos^{2021}(x)}}dx&=\int_0^{\frac{\pi}{2}}\frac{\cos^{2021}{(x)}}{\cos^{2021}{(x)}+\sin^{2021}(x)}dx\\
    2I&=\int_0^{\frac{\pi}{2}}\frac{\cos^{2021}{(x)}}{\cos^{2021}{(x)}+\sin^{2021}(x)}dx+\int_0^{\frac{\pi}{2}}\frac{\sin^{2021}{(x)}}{\cos^{2021}{(x)}+\sin^{2021}(x)}dx\\
    2I&=\int_0^{\frac{\pi}{2}}\frac{\cos^{2021}{(x)}+\sin^{2021}(x)}{\cos^{2021}{(x)}+\sin^{2021}(x)}dx\\
    I&=\frac{\pi}{4}
\end{align*}
\hfill $\square$
\end{document}