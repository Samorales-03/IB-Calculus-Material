\documentclass[spanish,12pt]{article}

\usepackage{enumitem}
\usepackage{tikz}
\usepackage{pgfplots}
\usepackage{amssymb}
\usepackage{booktabs}
\usepackage{enumerate}
\usepackage{multirow}
\usepackage{amsmath}
\usepackage{graphicx}
\usepackage[margin=1in]{geometry}
\usepackage{indentfirst}
\usepackage[spanish]{babel}
\usepackage[utf8]{inputenc}
\usepackage{fancyhdr}
\usepackage{tikz-feynman}
\usepackage{physics}
\usepackage{color}
\renewcommand{\baselinestretch}{1.5}
\newcommand{\dydx}{\frac{dy}{dx}}
\newcommand{\R}{\mathbb{R}}
\setlength{\parskip}{0.4cm}

\pagestyle{fancy}
\fancyhead{}
\fancyfoot{}
\usepackage{titlesec}
\titleformat{\section}
  {\normalfont\Large\bfseries}{\thesection}{1em}{}[{\titlerule[0.8pt]}]
\usepackage{hyperref}
\hypersetup{
    colorlinks=true,
    linkcolor=blue,
    filecolor=magenta,      
    urlcolor=cyan,
    pdftitle={IB Cálculo},
}
\fancyhead[l]{\fontsize{8}{12}\slshape\MakeUppercase{IB Cálculo}}
\fancyhead[R]{\slshape{S.Morales y J. Pulido}}
\fancyfoot[c]{\thepage}
\pgfplotsset{compat=1.17}
\begin{document}
	\begin{titlepage}
	\begin{center}
	\hspace{0pt}
	\vfill
	{\Large\textbf{{IB Cálculo}}}
	
	\medskip
	Clase 8: Integrales Definidas
	
	\medskip
    Samuel Morales y Julio Pulido
	
	\thispagestyle{empty}
	\vfill
	\end{center}
	\end{titlepage}
\newpage
\tableofcontents
\newpage
\section{Integración Definida}
\subsection{Teorema Fundamental del Cálculo}

$$\int^b_af(x)dx=F(b)-F(a)\iff F'(x)=f(x)$$

Para la prueba del teorema, les recomendamos ver el video.

\subsection{Propiedades de las integrales de definidas}

\begin{align*}
    \int^b_af(x)dx&=-\int^a_bf(x)dx\\
    \int^a_af(x)dx&=0\\
    \int^b_af(x)dx+\int^c_bf(x)dx&=\int^c_af(x)dx \quad \text{Para }a\leq b \leq c
\end{align*}

\subsection{Interpretación de la integral definida}

La integral de $a$ a $b$ de una función corresponde al área comprendida entre el eje $x$ y la función, entre los límites $a$ y $b$. En otras palabras, el área debajo de la curva. 

Cuando $f(x)$ tiene algún valor negativo en el intervalo $[a,b]$, el área entre la curva y el eje $x$ será:

$$\int^b_a|f(x)|dx$$

\textit{Ejemplo.}
Encuentre el área entre el eje $x$ y la curva para $f(x)=x^3-3x^2-x+3$ entre -2 y 4.
\begin{figure}[h!]
		\begin{center}
	\begin{tikzpicture}
\begin{axis}
[axis lines=center,]
\addplot[samples=100,domain=-2:4,color=blue,]{x*x*x-x*x-x*x-x*x-x+3};
\end{axis}
\end{tikzpicture}
	\caption{Gráfico de $f(x)=x^3-3x^2-x+3$}
	\end{center}
	\end{figure}
Primero averiguamos si segmentos de la función con valores negativos, y luego dividir la función en segmentos. 

Para esto, encontramos primero los ceros de la función.
\begin{align*}
   f(x)=x^3-3x^2-x+3&=0\\
   (x+1)(x-1)(x-3)&=0\\
   x=-1,1,3
\end{align*}
Con estos tres puntos, y los limites inferiores y superiores (-2 y 4 respectivamente), se pueden establecer 4 intervalos.

\begin{align*}
    Área&=\int^{4}_{-2}|f(x)|dx\\
    Área&=\abs{\int^{-1}_{-2}f(x)dx}+\abs{\int^{1}_{-1}f(x)dx}+\abs{\int^{3}_{1}f(x)dx}+\abs{\int^{4}_{3}f(x)dx}\\
    Área&=\abs{F(-1)-F({-2})}+\abs{F(1)-F(-1)}+\abs{F(3)-F(1)}+\abs{F(4)-F(3)}
\end{align*}
Encontramos la primitiva
\begin{align*}
    F(x)&=\int f(x)dx\\
    F(x)&=\int x^3-3x^2-x+3dx\\
    F(x)&=\frac{x^4}{4}-x^3-\frac{x^2}{2}+3x
\end{align*}

Sabiendo esto, el resto de el trabajo es tedioso pero trivial; meramente algebráico.
\section{Ejercicios}
\begin{enumerate}[1)]
    \item Resuelva las siguientes integrales definidas
    \begin{enumerate}
        \item $\int^5_0 x^2dx$ (I)
        \item $\int^{\ln{k}}_{0} e^xdx$ (II)
        \item $\int^{\frac{\pi}{2}}_{0} \sin{x}dx$ (I)
        \item $\int^{0}_{-1} 5(1-2x)^2$ (II)
    \end{enumerate}
    \item Determine el área entre
    \begin{enumerate}
        \item $x=0;\;x=6;\;y=0;\;y=x$ (I)
        \item $x=-2\pi;\;x=2\pi;\;y=0;\;y=\sin{x}$ (I)
        \item $y=x^2-4;\;y=3x$ (II)
        \item $x=-2\pi;\;x=2\pi;\;y=\sin{x};\;y=\cos{x}$ (II)
        \item $y=x+4;\;y=4-x;\;y=\frac{x^2}{4}-4$ (III)
        \item $x=0;\;x=k\pi;\;y=\sin{x}+1$ (III)
    
    \end{enumerate}
\end{enumerate}
\end{document}