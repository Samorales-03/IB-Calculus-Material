\documentclass[spanish,12pt]{article}

\usepackage{enumitem}
\usepackage{tikz}
\usepackage{pgfplots}
\usepackage{amssymb}
\usepackage{booktabs}
\usepackage{enumerate}
\usepackage{multirow}
\usepackage{amsmath}
\usepackage{graphicx}
\usepackage[margin=1in]{geometry}
\usepackage{indentfirst}
\usepackage[spanish]{babel}
\usepackage[utf8]{inputenc}
\usepackage{fancyhdr}
\usepackage{tikz-feynman}
\usepackage{physics}
\usepackage{color}
\renewcommand{\baselinestretch}{1.5}
\newcommand{\dydx}{\frac{dy}{dx}}
\newcommand{\R}{\mathbb{R}}
\setlength{\parskip}{0.4cm}

\pagestyle{fancy}
\fancyhead{}
\fancyfoot{}
\usepackage{titlesec}
\titleformat{\section}
  {\normalfont\Large\bfseries}{\thesection}{1em}{}[{\titlerule[0.8pt]}]
\usepackage{hyperref}
\hypersetup{
    colorlinks=true,
    linkcolor=blue,
    filecolor=magenta,      
    urlcolor=cyan,
    pdftitle={IB Cálculo},
}
\fancyhead[l]{\fontsize{8}{12}\slshape\MakeUppercase{IB Cálculo}}
\fancyhead[R]{\slshape{S.Morales y J. Pulido}}
\fancyfoot[c]{\thepage}
\pgfplotsset{compat=1.17}
\begin{document}
	\begin{titlepage}
	\begin{center}
	\hspace{0pt}
	\vfill
	{\Large\textbf{{IB Cálculo}}}
	
	\medskip
	Clase 11: Ejercicios Métodos de Integración
	
	\medskip
    Samuel Morales y Julio Pulido
	
	\thispagestyle{empty}
	\vfill
	\end{center}
	\end{titlepage}
\begin{enumerate}[1)]
    \item Encuentre las siguientes integrales indefinida utilizando el o los métodos de su conveniencia.
    \begin{enumerate}
        \item $\int \frac{1}{x^2-3x-4}dx$ (I)
        \item $\int xe^{x^2}dx$ (I)
        \item $\int (x^2+3)\sqrt{x^3+9x+4}dx$ (II)
        \item $\int \frac{4}{x^2+5x-14}dx$ (I)
        \item $\int \tan(x)dx$ (II)
        \item $\int \sin^2(x)+\cos^2(x)dx $ (I)
        \item $\int \ln{x}dx$ (II)
        \item $\int \frac{w^2+7}{(w+2)(w-1)(w-4)}dw$ (III)
    \end{enumerate}
    \item Encuentre las siguientes integrales definida utilizando el o los métodos de su conveniencia.
    \begin{enumerate}
        \item $\int_{0}^{2\pi} 3x^2\cos{x^3}dx$ (I)
        \item $\int_{0}^{1} \frac{8-3t}{10t^2+13t-3}dx$ (I)
        \item $\int_2^{5} \frac{x}{\sqrt{x-1}}dx$ (II)
        \item $\int_0^e x\ln{x}dx$ (II)
        \item $\int_0^1 \arccos{x}dx$ (II)
        \item $\int_{-1}^{1} \frac{1}{x^2+1}dx$ (II)
        \item $\int_{-1}^{1} \frac{1}{x^2+5}dx$ (II)
        \item $\int_{\frac{\pi}{4}}^{\frac{3\pi}{4}} \frac{x}{\sin^2(x)}dx$ (II)
    
    \end{enumerate}
    \item Responda cada inciso basándose en lo probado anteriormente.
    \begin{enumerate}
        \item Utilizando una sustitución, pruebe qué $\int_a^b f(x)dx=\int_a^b f(a+b-x)dx$ (III) (\textit{Pista:} sustituya por $u=a+b-x$ (II))
        \item Demuestre qué $\int_0^{\frac{\pi}{2}}\frac{\cos x}{\sin{x}+\cos{x}}=\frac{\pi}{4}$. Recuerde que $\cos{(\frac{\pi}{2}-\theta)}=\sin{\theta}$. 
        \item Resuelva $\int_0^{\frac{\pi}{2}}\frac{1}{1+\tan(x)}dx$
        \item Resuelva $\int_0^{\frac{\pi}{2}}\frac{1}{1+\tan^{2021}(x)}dx$
    \end{enumerate}
\end{enumerate}
\end{document}