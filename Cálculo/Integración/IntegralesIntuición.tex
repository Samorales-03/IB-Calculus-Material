\documentclass[spanish,12pt]{article}

\usepackage{enumitem}
\usepackage{tikz}
\usepackage{pgfplots}
\usepackage{amssymb}
\usepackage{booktabs}
\usepackage{enumerate}
\usepackage{multirow}
\usepackage{amsmath}
\usepackage{graphicx}
\usepackage[margin=1in]{geometry}
\usepackage{indentfirst}
\usepackage[spanish]{babel}
\usepackage[utf8]{inputenc}
\usepackage{fancyhdr}
\usepackage{tikz-feynman}
\usepackage{physics}
\usepackage{color}
\renewcommand{\baselinestretch}{1.5}
\newcommand{\dydx}{\frac{dy}{dx}}
\newcommand{\R}{\mathbb{R}}
\setlength{\parskip}{0.4cm}

\pagestyle{fancy}
\fancyhead{}
\fancyfoot{}
\usepackage{titlesec}
\titleformat{\section}
  {\normalfont\Large\bfseries}{\thesection}{1em}{}[{\titlerule[0.8pt]}]
\usepackage{hyperref}
\hypersetup{
    colorlinks=true,
    linkcolor=blue,
    filecolor=magenta,      
    urlcolor=cyan,
    pdftitle={IB Cálculo},
}
\fancyhead[l]{\fontsize{8}{12}\slshape\MakeUppercase{IB Cálculo}}
\fancyhead[R]{\slshape{S.Morales y J. Pulido}}
\fancyfoot[c]{\thepage}
\pgfplotsset{compat=1.17}
\begin{document}
	\begin{titlepage}
	\begin{center}
	\hspace{0pt}
	\vfill
	{\Large\textbf{{IB Cálculo}}}
	
	\medskip
	Clase 7: Integrales Indefinidas
	
	\medskip
    Samuel Morales y Julio Pulido
	
	\thispagestyle{empty}
	\vfill
	\end{center}
	\end{titlepage}
\newpage
\tableofcontents
\newpage
\section{Definición de integral indefinida}

La integral indefinida, tambien conocida como anti-derivada se define de la siguiente manera.

\textit{Definicion}. Sea $f(x)$ una funcion. Llamamos a la integral de $f(x)$ la antiderivada de $f(x)$, en otras palabras.

\begin{align*}
    &f:=\R \rightarrow \R\\
    &\int f(x) dx = F(x) \iff F'(x)=f(x)
\end{align*}

\textit{Ejemplos.}
\begin{enumerate}[(1)]
    \item Sea $f(x)=1$, halle $\int f(x) dx$
    
        Recordando que si $f(x)=x$, $f'(x)=1$. $F(x)=x+C$
        
\hfill $\square$

\item Sea $f(x)=\cos(x),$ halle $\int f(x)dx$
\item Resuelva $\int \frac{1}{x}+e^x dx$
\end{enumerate}

\section{Propiedades de integración}

(Comentar tabla de integrales directas en formulario)
Cosas a considerar:
\begin{align*}
    \int x^ndx &= \frac{x^{n+1}}{n+1}+C\\
    \int kf(x)dx &= k\int f(x)dx \qquad \text{Para } k\in\R\\
    \int f(x) + g(x)dx &= \int f(x)dx + \int g(x)dx\\
    \int f(x)g(x)dx &\neq \int f(x)dx\int g(x)dx
\end{align*}

\section{Ejercicios}
\begin{enumerate}[1)]
    \item Encuentre las sioguientes integrales
    \begin{enumerate}
        \item $\int x^2dx$ (I)
        \item $\int e^xdx$ (I)
        \item $\int a^xdx$ (III)
        \item $\int \sin{x}dx$ (I)
        \item $\int \cos{\theta}d\theta$ (I)
        \item $\int \sec^2{x}dx$ (I)
        \item $\int \frac{1}{x}dx$ (I)
        \item $\int x^{-n}dx$ (II)
        \item $\int 2x^2 + \sin{x} + 3 dx$ (I)
        \item $\int e^xdx$ (I)
        \item $\int e^{\sin{x}}dk$ (I)
    \end{enumerate}
    \item Determine la ecuación itinerario para un móvil de acuerdo a la aceleración (Pista: La derivada de la ecuación itinerario es la ecuación de velocidad y la derivada de la ecuación velocidad es la aceleración).
    \begin{enumerate}
        \item aceleración$=a$ (II)
        \item aceleración$=\sin(t)$ (III)
    \end{enumerate}
    \item Se define $f_n(g(x))$ como la enésima integral de $g(x)$. Determine $f_1(x), f_2(x), f_3(x), f_4(x), f_5(x)$, $f_{4n}(x), f_{4n+1}(x), f_{4n+2}(x)$ y $f_{4n+3}(x)$ para:
        \begin{enumerate}
            \item $g(x)=e^x$ (II)
            \item $g(x)=\sin{x}$ (III)
        \end{enumerate}
\end{enumerate}
\end{document}