\documentclass[spanish,12pt]{article}

\usepackage{enumitem}
\usepackage{tikz}
\usepackage{pgfplots}
\usepackage{amssymb}
\usepackage{booktabs}
\usepackage{multirow}
\usepackage{amsmath}
\usepackage{graphicx}
\usepackage[margin=1in]{geometry}
\usepackage{indentfirst}
\usepackage[spanish]{babel}
\usepackage[utf8]{inputenc}
\usepackage{fancyhdr}
\usepackage{tikz-feynman}
\usepackage{physics}
\usepackage{color}
\usepackage[sorting=none]{biblatex}
\addbibresource{Bibliography.bib}
\renewcommand{\baselinestretch}{1.5}
\setlength{\parskip}{0.4cm}

\pagestyle{fancy}
\fancyhead{}
\fancyfoot{}
\usepackage{titlesec}
\titleformat{\section}
  {\normalfont\Large\bfseries}{\thesection}{1em}{}[{\titlerule[0.8pt]}]
\usepackage{hyperref}
\hypersetup{
    colorlinks=true,
    linkcolor=blue,
    filecolor=magenta,      
    urlcolor=cyan,
    citecolor=blue,
    pdftitle={IB Cálculo},
}
\fancyhead[l]{\fontsize{8}{12}\slshape\MakeUppercase{IB Cálculo}}
\fancyhead[R]{\slshape{Samuel Morales y Julio Pulido}}
\fancyfoot[c]{\thepage}
\pgfplotsset{compat=1.17}

\begin{document}
	\begin{titlepage}
	\begin{center}
	\hspace{0pt}
	\vfill
	{\Large\textbf{{IB Cálculo}}}
	
	\medskip
	Clase 4: Ejercicios
	
	\medskip
    Samuel Morales y Julio Pulido
	
	\thispagestyle{empty}
	\vfill
	\end{center}
	\end{titlepage}
\newpage
\tableofcontents
\newpage
\section{Ejercicios de Límites}

Determine si las siguientes funciones son continuas en todo $x \in \mathbb{R}$
\begin{align*}
    f(x) &=\begin{cases} 
      2x+1 & x<2 \\
    3x-2 & x\geq2
   \end{cases}\\
   f(x) &=\begin{cases} 
      x^2-1 & x=2 \\
    2x-1 & x>2
   \end{cases}\\
   f(x)&= \frac{x+1}{\sqrt{x^2+1}}
\end{align*}

Calcule los siguientes límites
\begin{align*}
    &\lim_{x\to a} \frac{a^2x^2-b^2}{ax-b}\\
    &\lim_{x\to \infty}\frac{3x}{x+3}\\
    &\lim_{x\to 2} \frac{2}{2+\frac{2}{2-x}}\\
    &\lim_{x\to \infty} \frac{2x+3}{5x^2-2}
\end{align*}

Encuentre las asíntotas verticales y horizontales de las siguientes funciones
\begin{align*}
    f(x)&=\frac{3x}{6x-1}\\
    f(x)&=\frac{3x^3-x}{x^2}\\
    f(x)&=\frac{x^2+3}{3-x^2}
\end{align*}

\section{Ejercicios de Derivación por principios}

Encuentre, desde principios, la derivada de las siguientes funciones. Luego calcule la pendiente de la recta tangente en el punto indicado.

\begin{align*}
    f(x) &=\frac{x}{x+1} \quad x=0\\
    f(x) &=\frac{1}{x^2} \quad x=2
\end{align*}

Encuentre el valor de $x$ tal que 
\begin{align*}
    f(x)&=\frac{1}{x^2}\\
    f'(x)&=-\frac{1}{4}
\end{align*}

\section{Ejercicios de Derivación con reglas}
Sea $f^{(n)}(x)$ la enésima derivada de $f(x)=xe^{2x}$, demuestre que $f^{(n)}(x)= (2^nx+n2^{n-1})e^{2x}$ para todo $x \in \mathbb{N}$. (\textit{Pista}. Utilice el concepto de inducción matemática).

Sea $g(n)= \frac{1}{x}$, demuestre que $g^{(n)}(x)= (-1)^nn!x^{-(n+1)}$ 

Para todo $f(x)$, encuentre $f'(x)$
\begin{align*}
    f(x)&= \frac{3-2x^3+x^4}{x}\\
    f(x)&= \sqrt[5]{x^2}\\
    f(x)&= \left(\frac{x^2-1}{x}\right)^5
\end{align*}

Encuentre $f^{(k)}(x)$ de :
\begin{align*}
    f(x)&= e^x\\
    f(x)&= x^k\\
    f(x)&= \frac{1}{x^k}
\end{align*}

Calcule para las siguientes funciones $f'(x)$ evaluada en los puntos dados:
\begin{align*}
    f(x)&= \sqrt{2-\sqrt{x}}\quad\quad x=2\\
    f(x)&= (2-\sqrt{x})^3\quad\quad x=5\\
    f(x)&= \lfloor x^2\rfloor \quad\quad x\in (0,1)
\end{align*}

Calcule el o los valores de $x$ donde $f(x)$ y $g(x)$ tienen la misma pendiente
\begin{align*}
    &f(x)=ax\quad\quad g(x)=\frac{x^3}{3}\tag{¿Para que valores de $a$ pueden tener la misma pendiente?}\\
    &f(x)=\sin{x} \quad\quad g(x)=\cos{x}\tag{$x \in [0,2\pi]$}\\
    &f(x)=ax^2+c \quad\quad g(x)=bx^2+d \tag{$a\neq b\neq 0$}
\end{align*}

\end{document}