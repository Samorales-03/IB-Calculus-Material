\documentclass[spanish,12pt]{article}

\usepackage{enumitem}
\usepackage{tikz}
\usepackage{pgfplots}
\usepackage{amssymb}
\usepackage{booktabs}
\usepackage{multirow}
\usepackage{amsmath}
\usepackage{graphicx}
\usepackage[margin=1in]{geometry}
\usepackage{indentfirst}
\usepackage[spanish]{babel}
\usepackage[utf8]{inputenc}
\usepackage{fancyhdr}
\usepackage{tikz-feynman}
\usepackage{physics}
\usepackage{color}
\renewcommand{\baselinestretch}{1.5}
\setlength{\parskip}{0.4cm}

\pagestyle{fancy}
\fancyhead{}
\fancyfoot{}
\usepackage{titlesec}
\titleformat{\section}
  {\normalfont\Large\bfseries}{\thesection}{1em}{}[{\titlerule[0.8pt]}]
\usepackage{hyperref}
\hypersetup{
    colorlinks=true,
    linkcolor=blue,
    filecolor=magenta,      
    urlcolor=cyan,
    pdftitle={IB Cálculo},
}
\fancyhead[l]{\fontsize{8}{12}\slshape\MakeUppercase{IB Cálculo}}
\fancyhead[R]{\slshape{S.Morales y J. Pulido}}
\fancyfoot[c]{\thepage}
\pgfplotsset{compat=1.17}
\begin{document}
	\begin{titlepage}
	\begin{center}
	\hspace{0pt}
	\vfill
	{\Large\textbf{{IB Cálculo}}}
	
	\medskip
	Clase 1:Límites
	
	\medskip
    Samuel Morales y Julio Pulido
	
	\thispagestyle{empty}
	\vfill
	\end{center}
	\end{titlepage}
\newpage
\tableofcontents
\newpage
\section{Temas}
\begin{itemize}
    \item Definición
    \item Función continua en un punto y en un intervalo
    \item numero específico
    \item no converge en 0
    \item converge en infty
    \item no converge a infinito
    \item Indeterminado a infinito
\end{itemize}
\section{Ejemplos}
Es hacia donde tiende una función.

\begin{align*}
    \lim_{n\to x}f(n)=L \iff \lim_{n\to x^+}f(n)=\lim_{n\to x^-}f(n)=L
\end{align*}
\textit{Ejemplo.}
\begin{align*}
    \lim_{x\to 2} x^2=4\\
    \lim_{x\to 2^+} x^2=\lim_{x\to2^-}x^2=4
\end{align*}
\begin{figure}[h]
		\begin{center}
	\begin{tikzpicture}
		\begin{axis}[
			xlabel={$x$},
			ylabel={$f(x)$},
			xmin=0, xmax=5,
			ymin=0, ymax=10,
			legend pos=north west,
			ymajorgrids=true,
			axis lines=left,
			scaled y ticks=false,
			]
			
			\addplot[
			color=blue,
			domain=0:5
			]
		{x^2};
		\end{axis}
	\end{tikzpicture}
	\caption{Gráfico de $x^2$}
	\end{center}
	\end{figure}
	
\hfill $\square$
\newpage
\textit{Ejemplo}. ¿Cuál es el límite cuando $x$ tiende a 0 de $1\over x$?
    
    \begin{figure}[h!]
		\begin{center}
	\begin{tikzpicture}
		\begin{axis}[
			xlabel={$x$},
			ylabel={$f(x)$},
			xmin=-5, xmax=5,
			ymin=-10, ymax=10,
			legend pos=north west,
			ymajorgrids=true,
			axis lines=center,
			scaled y ticks=false,
			]
			
			\addplot[samples=100,
			color=blue,
			domain=-5:-0.001
			]
		{1/x};
		\addplot[samples=100,
			color=blue,
			domain=0.001:5
			]
		{1/x};
		\end{axis}
	\end{tikzpicture}
	\caption{Gráfico de $\frac{1}{x}$}
	\end{center}
	\end{figure}
	
	
	
	\textit{Ejemplo}. ¿Cuál es el límite cuando $x$ tiende a infinito de $e^{-x}$?
    
    \begin{figure}[h!]
		\begin{center}
	\begin{tikzpicture}
		\begin{axis}[
			xlabel={$x$},
			ylabel={$f(x)$},
			xmin=0, xmax=15,
			ymin=0, ymax=1,
			legend pos=north west,
			ymajorgrids=true,
			axis lines=center,
			scaled y ticks=false,
			]
			
			\addplot[
			color=blue,
			domain=0:15
			]
		{e^(-x)};
		\end{axis}
	\end{tikzpicture}
	\caption{Gráfico de $e^{-x}$}
	\end{center}
	\end{figure}
	$$\lim_{x\to\infty}e^{-x}=0$$
\textit{Ejemplo}. ¿Cuál es el límite cuando $x$ tiende a menos infinito ($-\infty$) de $f(x)=x$?
    
    \begin{figure}[h!]
		\begin{center}
	\begin{tikzpicture}
		\begin{axis}[
			xlabel={$x$},
			ylabel={$f(x)$},
			xmin=-15, xmax=15,
			ymin=-15, ymax=15,
			legend pos=north west,
			ymajorgrids=true,
			axis lines=center,
			scaled y ticks=false,
			]
			
			\addplot[
			color=blue,
			domain=-15:15
			]
		{x};
		\end{axis}
	\end{tikzpicture}
	\caption{Gráfico de $x$}
	\end{center}
	\end{figure}
	$$\lim_{x\to-\infty}x=-\infty$$	
	\textit{Ejemplo}. ¿Cuál es el límite cuando $x$ tiende a infinito de $f(x)=\sin{x}$?
    
    \begin{figure}[h!]
		\begin{center}
	\begin{tikzpicture}
\begin{axis}
[axis lines=center,]
\addplot[samples=500,domain=-4*pi:4*pi,color=blue,]{sin(deg(x))};
\end{axis}
\end{tikzpicture}
	\caption{Gráfico de $\sin{x}$}
	\end{center}
	\end{figure}
	$$\lim_{x\to\infty}\sin{x}=L\quad\quad\text{donde}\quad\quad L\in [-1,1]$$
	(indeterminado)
	\newpage
\section{Ejercicios}
Determine si los siguientes límites convergen. Si es así, determine su valor.
\begin{align*}
&\lim_{x\to-2}\frac{x^2-4}{x+2}\\
&\lim_{x\to2}\lfloor x\rfloor\\
&\lim_{x\to\infty} \frac{x+1}{x+10^{1000000}}
\end{align*}
$$\lim_{x\to2} \begin{cases} 
      x-3 & x<2 \\
    x+1 & x\geq2
   \end{cases}
$$

\textit{Desafio}. Pruebe que 
\begin{align*}
    \sum_{n=0}^\infty ax^n=\frac{a}{1-x} \quad\quad \forall \abs{x}<1
\end{align*}
\end{document}