\documentclass[spanish,12pt]{article}

\usepackage{enumitem}
\usepackage{tikz}
\usepackage{pgfplots}
\usepackage{amssymb}
\usepackage{booktabs}
\usepackage{multirow}
\usepackage{amsmath}
\usepackage{graphicx}
\usepackage[margin=1in]{geometry}
\usepackage{indentfirst}
\usepackage[spanish]{babel}
\usepackage[utf8]{inputenc}
\usepackage{fancyhdr}
\usepackage{tikz-feynman}
\usepackage{physics}
\usepackage{color}
\renewcommand{\baselinestretch}{1.5}
\setlength{\parskip}{0.4cm}

\pagestyle{fancy}
\fancyhead{}
\fancyfoot{}
\usepackage{titlesec}
\titleformat{\section}
  {\normalfont\Large\bfseries}{\thesection}{1em}{}[{\titlerule[0.8pt]}]
\usepackage{hyperref}
\hypersetup{
    colorlinks=true,
    linkcolor=blue,
    filecolor=magenta,      
    urlcolor=cyan,
    pdftitle={IB Cálculo},
}
\fancyhead[l]{\fontsize{8}{12}\slshape\MakeUppercase{IB Cálculo}}
\fancyhead[R]{\slshape{S.Morales y J. Pulido}}
\fancyfoot[c]{\thepage}
\pgfplotsset{compat=1.17}
\begin{document}
	\begin{titlepage}
	\begin{center}
	\hspace{0pt}
	\vfill
	{\Large\textbf{{IB Cálculo}}}
	
	\medskip
	Clase 2: Derivadas (Principios)
	
	\medskip
    Samuel Morales y Julio Pulido
	
	\thispagestyle{empty}
	\vfill
	\end{center}
	\end{titlepage}
\newpage
\tableofcontents
\newpage
\section{Temas}
\begin{itemize}
    \item Concepto de pendiente (entre valores discretos)
    \item Derivación por principio (Límite) 
\end{itemize}
\section{Pendiente de una curva}
\subsection{Derivando la ecuación de una pendiente}
\begin{figure}[h]
		\begin{center}
	\begin{tikzpicture}
		\begin{axis}[
			xlabel={$x$},
			ylabel={$f(x)$},
			xmin=0, xmax=3,
			ymin=0, ymax=5,
			xtick = {0,1,2,3},
			legend pos=north west,
			ymajorgrids=true,
			axis lines=left,
			scaled y ticks=false,
			]
			
			\addplot[
			color=red,
			samples=300,
			]
		{x^2};
		    \addplot[color=blue,]{3*(x)-2};
		\end{axis}
	\end{tikzpicture}
	\caption{Gráfico de $x^2$ y la pendiente entre 1 y 2}
	\end{center}
	\end{figure}
	
	\begin{align*}
	m&=\frac{f(2)-f(1)}{2-1}\\
	m&=\frac{4-1}{1}\\
	m&=3.
	\end{align*}

De manera mas general, la pendiente dentro dos puntos $x_2$ y $x_2$ es 
\begin{align*}
    m&=\frac{f(x_2)-f(x_1)}{x_2-x_1}\\
    m&=\frac{f(x_2)-f(x_1)}{\Delta x}.
\end{align*}
	\subsection{Derivación por Principio}
	\begin{align*}
	    f'(x)&=\lim_{h\to0}{\frac{f(x+h)-f(x)}{x+h-x}}\\
	    f'(x)&=\lim_{h\to0}{\frac{f(x+h)-f(x)}{h}}\\
	\end{align*}

\section{Ejemplos}
  \textit{Ejemplo.}  Si $f(x)=x$, calcule $f'(x)$
    \begin{align*}
        f'(x)&=\lim_{h\to0}\frac{f(x+h)-f(x)}{h}\\
        f'(x)&=\lim_{h\to0}\frac{x+h-x}{h}\\
        f'(x)&=\lim_{h\to0}\frac{h}{h}\\
        f'(x)&=\lim_{h\to0}1=1.
    \end{align*}
    \hfill $\square$
    
    \textit{Ejemplo.} $f(x)=x^2$
    \begin{align*}
	    f'(x)&=\lim_{h\to0}\frac{(x+h)^2-x^2}{h}\\
	    f'(x)&=\lim_{h\to0}{\frac{2hx+h^2}{h}}\\
	    f'(x)&=\lim_{h\to0}{2x+h}\\
	    f'(x)&=2x
	\end{align*}
\section{Ejercicios}
Para todo $f(x)$, encuentre, desde los principios, $f'(x)$
\begin{align*}
    f(x)&=x^3+x^2+x+1\\
    f(x)&=x^2-2\\
    f(x)&=x^2-5\\
    f(x)&=x^2+4\\
    f(x)&=x^2+n\\
    f(x)&=ax \quad \quad a\in \mathbb{R}\\
    f(x)&=\sqrt{x}\\
    f(x)&=\frac{1}{x}\\
    f(x)&=a
\end{align*}
\textit{Desafío}. Conociendo que un móvil de mueve siguiendo la siguiente ecuación:
$$x(t)=x_0+v_0t+\frac{1}{2}at^2$$

En donde $x(t)$ es la posición en un tiempo $t$, $x_0$ la posición inicial, $v_0$ la velocidad inicial y $a$ la aceleración.

Calcule la aceleración del objeto utilizando la derivación desde principios

\end{document}