\documentclass[spanish,12pt]{article}

\usepackage{enumitem}
\usepackage{tikz}
\usepackage{pgfplots}
\usepackage{amssymb}
\usepackage{booktabs}
\usepackage{multirow}
\usepackage{amsmath}
\usepackage{graphicx}
\usepackage[margin=1in]{geometry}
\usepackage{indentfirst}
\usepackage[spanish]{babel}
\usepackage[utf8]{inputenc}
\usepackage{fancyhdr}
\usepackage{tikz-feynman}
\usepackage{physics}
\usepackage{color}
\renewcommand{\baselinestretch}{1.5}
\setlength{\parskip}{0.4cm}

\pagestyle{fancy}
\fancyhead{}
\fancyfoot{}
\usepackage{titlesec}
\titleformat{\section}
  {\normalfont\Large\bfseries}{\thesection}{1em}{}[{\titlerule[0.8pt]}]
\usepackage{hyperref}
\hypersetup{
    colorlinks=true,
    linkcolor=blue,
    filecolor=magenta,      
    urlcolor=cyan,
    pdftitle={IB Cálculo},
}
\fancyhead[l]{\fontsize{8}{12}\slshape\MakeUppercase{IB Cálculo}}
\fancyhead[R]{\slshape{S.Morales y J. Pulido}}
\fancyfoot[c]{\thepage}
\pgfplotsset{compat=1.17}
\begin{document}
	\begin{titlepage}
	\begin{center}
	\hspace{0pt}
	\vfill
	{\Large\textbf{{IB Cálculo}}}
	
	\medskip
	Clase 3: Derivadas (Reglas)
	
	\medskip
    Samuel Morales y Julio Pulido
	
	\thispagestyle{empty}
	\vfill
	\end{center}
	\end{titlepage}
\newpage
\tableofcontents
\newpage
\section{Temas}
\begin{itemize}
    \item Reglas de derivación
    \item 
    \item 
    \item 
    \item 
\end{itemize}
\section{Reglas de derivación}
\subsection{Potencia}
\begin{align*}
    f(x)&=x^n \quad \quad \forall n\in \mathbb{R}\\
    \implies f'(x)&=nx^{n-1}
\end{align*}

\textit{Prueba}.
\begin{align*}
    f'(x)&=\lim_{h\to0}{\frac{(x+h)^n-x^n}{h}}\\
    f'(x)&=\lim_{h\to0}{\frac{\sum\limits_{k=0}^{n}{\binom{n}{n-k}}x^{n-k}h^{k}-x^n}{h}}\\
    f'(x)&=\lim_{h\to0}{\frac{x^n+\sum\limits_{k=1}^{n}{\binom{n}{n-k}}x^{n-k}h^{k}-x^n}{h}}\\
    f'(x)&=\lim_{h\to0}{\frac{h\sum\limits_{k=1}^{n}{\binom{n}{n-k}}x^{n-k}h^{k-1}}{h}}\\
    f'(x)&=\lim_{h\to0}{\sum\limits_{k=1}^{n}{\binom{n}{n-k}}x^{n-k}h^{k-1}}\\
    f'(x)&=\binom{n}{n-1}x^{n-1}\\
    f'(x)&=nx^{n-1}\\
\end{align*}
\hfill $\square$
\subsection{Suma}
    \begin{align*}
        h(x)&=f(x)+g(x)\\
        h'(x)&=f'(x)+g'(x)
    \end{align*}
\subsection{Producto}
\begin{align*}
    h(x)&=f(x)g(x)\\
    h'(x)&=g'(x)f(x)+f(x)g'(x)
\end{align*}
\textit{Prueba}.
\begin{align*}
    h(x)&=g(x)f(x)\\
    h'(x)&=\lim_{h\to0}\frac{g(x+h)f(x+h)-g(x)f(x)}{h}\\
    h'(x)&=\lim_{h\to0}\frac{g(x+h)f(x+h)-g(x)f(x)+{\color{red}g(x+h)f(x)-g(x+h)f(x)}}{h}\\
    h'(x)&=\lim_{h\to0}\frac{g(x+h)(f(x+h)-f(x))+f(x)(g(x+h)-g(x))}{h}\\
    h'(x)&=\lim_{h\to0}\frac{g(x+h)(f(x+h)-f(x))}{h}+\frac{f(x)(g(x+h)-g(x))}{h}\\
    h'(x)&=\lim_{h\to0}\frac{g(x+h)(f(x+h)-f(x))}{h}+\lim_{h\to0}\frac{f(x)(g(x+h)-g(x))}{h}\\
    h'(x)&=\lim_{h\to0}g(x+h)\lim_{h\to0}\frac{f(x+h)-f(x)}{h}+f(x)\lim_{h\to0}\frac{g(x+h)-g(x)}{h}\\
    h'(x)&=g(x)f'(x)+f(x)g'(x)
\end{align*}
\hfill $\square$
\subsection{Cadena}
$$(f(g(x)))'=f'(g(x))g'(x)$$
\subsection{Cociente}
$$h(x)=\frac{f(x)}{g(x)}$$
\begin{align}
    h'(x)=\frac{f'(x)g(x)-f(x)g'(x)}{(g(x))^2}\label{eq:1}
\end{align}
\newpage
\section{Derivadas estándar}

\begin{table}[h!]
    \centering
    \begin{tabular}{|cc|}
     \toprule
         f(x)&f'(x)  \\
    \midrule    
        $\sin{x}$ &  $\cos{x}$\\
        $\cos{x}$& $-\sin{x}$\\
        $e^x$&$e^x$\\
        $\ln{x}$& $\frac{1}{x}$\\
        $\tan{x}$& $\sec^2{x}$\\
        $\sec{x}$& $\sec{x}\tan{x}$\\
        $\cosec{x}$& $-\cosec{x}\cot{x}$\\
        $a^x$&$a^x\ln{a}$\\
        $\log_a{x}$&$\frac{1}{x\ln(a)}$\\
        $\arcsin{x}$& $\frac{1}{\sqrt{1-x^2}}$\\
        $\arccos{x}$&$-\frac{1}{\sqrt{1-x^2}}$\\
        $\arctan{x}$&$\frac{1}{\sqrt{1+x^2}}$\\
    \bottomrule
    \end{tabular}
    \caption{Derivadas directas}
    \label{tab:my_label}
\end{table}
\section{Ejercicios}
Encuentre la derivada de las siguientes funciones:
\begin{align*}
    f(x)&=x^5+3x^3-x+4\\
    f(x)&=\frac{1}{x^3}\\
    f(x)&=\sqrt[4]{x^3}\\
    f(x)&=\frac{4}{\sqrt{x^3}}\\
    f(x)&=\sin{(\cos{x})}\\
    f(x)&=(x^3+3)^3
\end{align*}
\textit{Desafío.} A partir de lo visto (regla del producto y regla de potencia), pruebe la regla del cociente.

\section{Regla de L'Hopital}

Para los limites que se indeterminen en casos como $\frac{0}{0} , \frac{\infty}{\infty}$. La regla de L'Hopital establece lo siguiente
\begin{align*}
    \lim\limits_{x\to a} \frac{f(x)}{g(x)}=\frac{f'(x)}{g'(x)}
\end{align*}
\section{Ejemplos}
\begin{align*}
    &\lim_{x\to\infty} \frac{x+1}{x+10^{1000000}}\\
    &\lim_{x\to\infty} \frac{x}{x}\\
    &\lim_{x\to0} \frac{x}{x}\\
    &\lim_{x\to\infty} \frac{x^3+1}{x}\\
    &\lim_{x\to1} \frac{x^2-1}{x^2-x}\\
    &\lim_{x\to\infty} \frac{x}{x^2-x}\\
     &\lim_{x\to0} \frac{x}{x^2-x}\\
    &\lim_{x\to\infty} \frac{\ln{x}}{x}\\
    &\lim_{x\to 0} x^x\\
    &\lim_{x\to 0} x^{(x^x)}\\
\end{align*}
\end{document}