\documentclass[spanish,12pt]{article}

\usepackage{enumitem}
\usepackage{tikz}
\usepackage{pgfplots}
\usepackage{amssymb}
\usepackage{booktabs}
\usepackage{enumerate}
\usepackage{multirow}
\usepackage{amsmath}
\usepackage{graphicx}
\usepackage[margin=1in]{geometry}
\usepackage{indentfirst}
\usepackage[spanish]{babel}
\usepackage[utf8]{inputenc}
\usepackage{fancyhdr}
\usepackage{tikz-feynman}
\usepackage{physics}
\usepackage{color}
\renewcommand{\baselinestretch}{1.5}
\setlength{\parskip}{0.4cm}

\pagestyle{fancy}
\fancyhead{}
\fancyfoot{}
\usepackage{titlesec}
\titleformat{\section}
  {\normalfont\Large\bfseries}{\thesection}{1em}{}[{\titlerule[0.8pt]}]
\usepackage{hyperref}
\hypersetup{
    colorlinks=true,
    linkcolor=blue,
    filecolor=magenta,      
    urlcolor=cyan,
    pdftitle={IB Cálculo},
}
\fancyhead[l]{\fontsize{8}{12}\slshape\MakeUppercase{IB Cálculo}}
\fancyhead[R]{\slshape{S.Morales y J. Pulido}}
\fancyfoot[c]{\thepage}
\pgfplotsset{compat=1.17}
\begin{document}
	\begin{titlepage}
	\begin{center}
	\hspace{0pt}
	\vfill
	{\Large\textbf{{IB Cálculo}}}
	
	\medskip
	Clase 5: Aplicación de Derivadas
	
	\medskip
    Samuel Morales y Julio Pulido
	
	\thispagestyle{empty}
	\vfill
	\end{center}
	\end{titlepage}
\newpage
\tableofcontents
\newpage
\begin{itemize}
    \item Interpretación
    \item Recta Tangente
    \item Recta Normal
\end{itemize}
\section{Interpretación de la derivada}

La derivada de $f(x)$ se entiende como la pendiente inmediata evaluada en un punto $x$. Si por ejemplo la derivada es un valor constante, quiere decir que la pendiente es constante. Si depende también de $x$, entonces la pendiente va a ir cambiando.

\section{Rectas tangente y normal}
\subsection{Recta tangente}
La tangente a un punto $x_1,y_1$ de una función $f(x)$ corresponde a la recta (lineal) que pasa por dicho punto y cuya pendiente corresponde a la pendiente de la función en ese punto.

Recordemos que la pendiente ($m$) está dada por

$$m=\frac{y_2-y_1}{x_2-x_1}$$

Entonces, utilizando la definición de derivada que conocemos, $m=f'(x_1)$ y asumiendo $y_2=y \land x_2=x$
\begin{align*}
    f'(x_1)&=\frac{y-y_1}{x-x_1}\\
    y&=f'(x_1)x-f'(x_1)x_1+y_1
\end{align*}

\begin{align*}
    t:=y(x)&=mx+n\\
    y(x)&=\underbrace{f'(x_1)}_{m}x+\underbrace{f(x_1)-x_1f'(x_1)}_{n}\\
\end{align*}

Se aclara que la intercepción $(n)$ se obtiene de el hecho de que la recta tangente pasa por el punto $(x_1,y_1)$, entonces:

\begin{align*}
    f(x_1)&=y(x_1)\\
    f(x_1)&=f'(x_1)x_1+n\\
    n&=f(x_1)-f'(x_1)x_1
\end{align*}

\hfill $\square$

\subsection{Recta Normal}

La normal a un punto $x_1,y_1$ de una función $f(x)$ corresponde a la recta (lineal) perpendicular a la tangente, y que pasa por dicho punto.

Recordemos que 

\begin{align*}
    m_1 \perp m_2 \iff m_1m_2=-1
\end{align*}

Entonces, si la recta tangente es de pendiente $m$, la recta normal tendrá que tener pendiente $-1\over m$. Reemplazando en la ecuación de la recta. 
$$n(x)=\frac{-1}{f'(x_1)}x+f(x_1)+x_1\frac{1}{f'(x_1)}$$


\subsection{Ejercicios}
\begin{enumerate}[1)]
    \item Encuentre la recta tangente a las siguientes funciones para x=1:
\begin{align*}
    f(x)&=2x^2-x+1\tag{I\footnotemark}\\
    f(x)&=\ln(x)\tag{I}\\
\end{align*}
\footnotetext{{Nuevo sistema de dificultad. I puede aparecer en la prueba como ejercicio sencillo. II puede aparecer en la prueba como ejercicio complejo. III no aparecerá en la prueba}}
    \item  Encuentre los puntos de la curva $y=\frac{1}{2-x}$ donde la pendiente es 1, y deduzca las rectas tangentes y normales en estos puntos (II).
    \item Generalice la función de las rectas tangentes a la función $\sin(x)$ para todos los puntos donde la pendiente sea $k$ (III).
    \item Para las siguientes funciones, encuentre las coordinadas de los puntos donde la recta tangente sea paralela al eje $x$:
\begin{align*}
    f(x)&=2x^3+3\tag{I}\\
    f(x)&=\ln{x}-x\tag{II}
\end{align*}
\end{enumerate}
\section{Optimización}
\subsection{Optimización e interpretación de la derivada}

Sabemos que la derivada representa la pendiente en un punto de una función, por lo tanto si en un punto la derivada es 0 nos encontramos ante un máximo o un mínimo (local o global). (ver gráfico) o un punto de inflexión.

El concepto de optimización apunta a la maximización o minimización de una función y tiene grandes utilidades prácticas. (Por ejemplo, minimizar costos o maximizar ganancias).

Para distinguir un máximo de un mínimo, podemos comprobar con valores cercanos al punto examinado. Por ejemplo:

\textit{Ejemplo.} ¿Dónde se maximiza $f(x)=-x^2+x$?

Debemos buscar el punto donde la derivada es 0, entonces:

\begin{align*}
    0&=f'(x)\\
    0&=-2x+1\\
    x&=\frac{1}{2}
\end{align*}

Para comprobar que este es un máximo, podemos considerar valores cercanos a $x=\frac{1}{2}$, por ejemplo, 1 y 0.

\begin{align*}
    f(0)&=0\\
    f\left(\frac{1}{2}\right)&=\frac{1}{4}\\
    f(1)=0
\end{align*}

Como los valores cercanos a $f(1/2)$ son menores que este, dicho punto será un máximo.

\subsection{Segunda derivada y puntos de inflexión}
La segunda derivada (denotada como $f''(x)\text{ o }\ f^{(2)}(x)$) corresponde a la derivada de la derivada de la función $f(x)$.
$$f''(x)=\frac{d}{dx}f'(x)$$

La mayor importancia de la segunda derivada radica en su interpretación. Se pueden extraer dos cosas:

(I) Como vimos anteriormente, cuando la derivada de una función es 0, ocurrirá un mínimo, un máximo o un punto de inflexión. 

\begin{itemize}
    \item Si $f'(x_1)=0 \;\land\; f''(x_1)<0$ en $x_1$ ocurre un máximo. 
    \item Si $f'(x_1)=0 \;\land\; f''(x_1)>0$ en $x_1$ ocurre un mínimo. 
    \item Si $f'(x_1)=0 \;\land\; f''(x_1)=0$ en $x_1$ ocurre un cambio de concavidad y un punto de inflexión. 
\end{itemize}

(II) Independiente del valor de $f'(x)$, cuando $f''(x)=0$ ocurre un punto de inflexión y un cambio de concavidad.

\begin{itemize}
    \item Si $f''(x)<0$ la función es cóncava hacia abajo
    \item Si $f''(x)>0$ la función es cóncava hacia arriba
\end{itemize}
\newpage
\subsection{Ejercicios}
\begin{enumerate}[1)]
    \item  Para las siguientes funciones encuentre (I)
    \begin{itemize}
    \item Puntos de inflexión
    \item Intervalos donde es cóncava hacia arriba
    \item intervalos donde es cóncava hacia abajo
    \end{itemize}
    \begin{enumerate}
        \item $f(x)=x^3-x$
        \item $f(x)=x^4-3x+2$
        \item $f(x)=x^4+x^3+x^2+x+1$ 
    \end{enumerate}
    \item Realice ejercicios 1,2,3,4,5,7,9,12 en las páginas 281-282 del libro (II).
    \item Realice ejercicios 1,2,3,4 en las páginas 283-284 del libro (II).
    \item Realice ejercicios 1,3,4,7 en la página 287 del libro (II).
    \item Un rectángulo se compone de lados $a$ y $b$ y suma un perímetro fíjo $P$. Encuentre la relación entre $a$ y $b$ (relación dada por $\frac{a}{b}$) tal que se maximice el área del rectángulo.  (III)
\end{enumerate}
\newpage
Ejercicio de sección B Paper 1 (16 puntos en una prueba de 120):

A continuación se muestra el gráfico de la función 
$$f(x)=\frac{x+1}{x^2+1}$$

\begin{figure}[h!]
		\begin{center}
	\begin{tikzpicture}
    \begin{axis}[
	ytick = {1},
    axis lines=center,]
    \addplot[
    samples=500,
    domain=-5:6,
    color=blue,
    ]
    {(x+1)/(x^2+1)};
    \end{axis}
\end{tikzpicture}
	        \caption{Gráfico de $\frac{x+1}{x^2+1}$}
	    \end{center}
	\end{figure}
   \begin{enumerate}[(a)]
       \item Halle $f'(x)$
       \item A partir de lo anterior, halle las coordenadas $x$ de los puntos en los que la pendiente del
gráfico de $f$ es igual a cero.
        \item Halle $f''(x)$, expresando la respuesta de forma $\frac{p(x)}{(x^2+1)^3}$, donde $p(x)$ es un polinomio de grado 3.

El punto (1,1) es un punto de inflexión. Hay otros dos puntos de inflexión.
        \item Halle las coordenadas $x$ de los otros dos puntos de inflexión.

   \end{enumerate} 
\end{document}