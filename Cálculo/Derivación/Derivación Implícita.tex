\documentclass[spanish,12pt]{article}

\usepackage{enumitem}
\usepackage{tikz}
\usepackage{pgfplots}
\usepackage{amssymb}
\usepackage{booktabs}
\usepackage{enumerate}
\usepackage{multirow}
\usepackage{amsmath}
\usepackage{graphicx}
\usepackage[margin=1in]{geometry}
\usepackage{indentfirst}
\usepackage[spanish]{babel}
\usepackage[utf8]{inputenc}
\usepackage{fancyhdr}
\usepackage{tikz-feynman}
\usepackage{physics}
\usepackage{color}
\renewcommand{\baselinestretch}{1.5}
\newcommand{\dydx}{\frac{dy}{dx}}
\newcommand{\R}{\mathbb{R}}
\setlength{\parskip}{0.4cm}

\pagestyle{fancy}
\fancyhead{}
\fancyfoot{}
\usepackage{titlesec}
\titleformat{\section}
  {\normalfont\Large\bfseries}{\thesection}{1em}{}[{\titlerule[0.8pt]}]
\usepackage{hyperref}
\hypersetup{
    colorlinks=true,
    linkcolor=blue,
    filecolor=magenta,      
    urlcolor=cyan,
    pdftitle={IB Cálculo},
}
\fancyhead[l]{\fontsize{8}{12}\slshape\MakeUppercase{IB Cálculo}}
\fancyhead[R]{\slshape{S.Morales y J. Pulido}}
\fancyfoot[c]{\thepage}
\pgfplotsset{compat=1.17}
\begin{document}
	\begin{titlepage}
	\begin{center}
	\hspace{0pt}
	\vfill
	{\Large\textbf{{IB Cálculo}}}
	
	\medskip
	Clase 6: Derivación Implícita
	
	\medskip
    Samuel Morales y Julio Pulido
	
	\thispagestyle{empty}
	\vfill
	\end{center}
	\end{titlepage}
\newpage
\tableofcontents
\newpage
\section{Derivación implícita}

Nótese que para una función $f: \mathbb{R}\rightarrow \mathbb{R}$. $f'(x)=\frac{dy}{dx}=y'$

Entendamos lo que es la derivación implicita a través de un ejemplo.

Dado que conocemos lo siguiente:
\begin{align*}
    f(x)&=e^x\\
    f'(x)&=e^x
\end{align*}

Calculemos la siguiente derivada con respecto a $x$

\begin{align*}
    h(x)=\ln{x}
\end{align*}

Simplificando un poco la notación.
\begin{align*}
    y&=\ln{x}\\
    e^y&=x
\end{align*}
Ahora, entendiendo que $y$ es una variable que depende de $x$, podemos derivar usando la regla de la cadena.

\begin{align*}
    e^y y'&=1\\
    y'&=\frac{1}{e^y}=\frac{1}{x}
\end{align*}
\hfill $\square$


De manera general, dado el siguiente sistema


$$y&=f^{-1}(x)$$


Se puede calcular la derivada de la inversa de esta función de la siguiente forma

\begin{align*}
    y&=f^{-1}(x)\\
    f(y)&=x\\
    f'(y)y'&=1\\
    y'&=\frac{1}{f'(y)}
\end{align*}

La derivación implícita es una aplicación de la regla de la cadena para poder derivar en función a $x$ una función que está dada en términos dependientes de $y$.


$$\frac{d}{dx}g(y)=g'(y)\frac{dy}{dx}=g'(y)y'$$


Desde aquí se puede obtener $y'$ despejando.

Veamoslo en otro ejemplo:

La ecuación de un círculo de radio $a$ está dada por

$$a^2=x^2+y^2$$

Para hallar la derivada de $y$ en función de $x$ en un punto usamos derivación implícita

\begin{align*}
    0&=2x+2yy'\\
    y'&=-\frac{x}{y}
\end{align*}

Ahora, para escribir la derivada como una función independiente de $y$ podemos despejar $y$ en la función inicial y sustituir

\begin{align*}
    a^2&=x^2+y^2\\
    y&=\sqrt{a^2-x^2}\\
    y'&=-\frac{x}{y}\\
    y'&=-\frac{x}{\sqrt{a^2-x^2}}
\end{align*}

Esta herramienta es especialmente útil para funciones inversas y para casos donde conozco el valor de $y$ y el despeje de la variable implique mayor complejidad.

\section{Ejercicios}
\begin{enumerate}[1)]
    \item Encuentre $\frac{dy}{dx}$ para las siguientes funciones (I)
    \begin{enumerate}
        \item $2y^2-3x^2=1$
        \item $(x+y)^2=2-3y$
        \item $x=\sqrt{2x^2-6y^3}$
        \item $2x^2=\frac{x+y}{x-y}$
    \end{enumerate}
    \item Encuentre la tangente y la normal a las siguientes curvas en el punto (1,1) (II)
    \begin{enumerate}
        \item $\frac{1}{x^3}+\frac{1}{y^3}=2$
        \item $\frac{1}{x+1}+\frac{1}{y+1}=1$
    \end{enumerate}
    \item Encuentre las coordenadas de los puntos donde la gradiente es 0 para $x^2+y^2=6x+8y$ (II)
    \item Dada la curva $x+y=x^2-2xy-y^2$:
    \begin{enumerate}
        \item Encuentre $\dydx$ (I)
        \item Muestre que $1-\dydx=\frac{2}{2x-2y+1}$ (II)
        \item Muestre que $\frac{d^2y}{dx^2}=\left(1-\dydx\right)$ (II, difícil)
    \end{enumerate}
\end{enumerate}
\end{document}