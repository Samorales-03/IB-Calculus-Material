\documentclass[spanish,12pt]{article}

\usepackage{enumitem}
\usepackage{tikz}
\usepackage{pgfplots}
\usepackage{amssymb}
\usepackage{booktabs}
\usepackage{enumerate}
\usepackage{multirow}
\usepackage{amsmath}
\usepackage{graphicx}
\usepackage[margin=1in]{geometry}
\usepackage{indentfirst}
\usepackage[spanish]{babel}
\usepackage[utf8]{inputenc}
\usepackage{fancyhdr}
\usepackage{tikz-feynman}
\usepackage{physics}
\usepackage{color}
\renewcommand{\baselinestretch}{1.5}

\newcommand{\dydx}{\frac{dy}{dx}}
\newcommand{\ddx}{\frac{dy}{dx}}
\newcommand{\R}{\mathbb{R}}
\newcommand{\N}{\mathbb{N}}
\newcommand*\Eval[3]{\left.#1\right\rvert_{#2}^{#3}}

\setlength{\parskip}{0.4cm}

\pagestyle{fancy}
\fancyhead{}
\fancyfoot{}
\usepackage{titlesec}
\titleformat{\section}
  {\normalfont\Large\bfseries}{\thesection}{1em}{}[{\titlerule[0.8pt]}]
\usepackage{hyperref}
\hypersetup{
    colorlinks=true,
    linkcolor=blue,
    filecolor=magenta,      
    urlcolor=cyan,
    pdftitle={Solucionario P3},
}
\fancyhead[l]{\fontsize{8}{12}\slshape\MakeUppercase{Métodos de prueba}}
\fancyhead[R]{\slshape{J. Pulido}}
\fancyfoot[c]{\thepage}
\pgfplotsset{compat=1.17}
\begin{document}
	\begin{titlepage}
	\begin{center}
	\hspace{0pt}
	\vfill
	Álgebra – I
	\medskip
	
	{\Large\textbf{{Clase 1: Métodos de prueba}}}
	
	\medskip
	Julio Pulido
	\thispagestyle{empty}
	\vfill
	\end{center}
	\end{titlepage}
\newpage
\section{Prueba matemática}
Una prueba consiste de una serie de pasos lógicos que validan un postulado.
\subsection{Prueba directa}
Consiste en probar un postulado a través de pasos algebraicos y/u otros postulados verdaderos.
\textbf{Pasos:}
\begin{enumerate}
    \item Identificar el postulado
    \item Convertirlo a lenguaje matemático si es que no lo está
    \item Usar álgebra o teoremas para deducir la validez del postulado
\end{enumerate}
\newpage
\subsection{Contraejemplo}
No es un tipo de prueba. Es una forma de determinar que un postulado es falso. Consiste en dar un ejemplo donde no se cumpla el postulado dado.
\subsection{Prueba por contradicción}
\textbf{Pasos:}
\begin{enumerate}
    \item Identificar el postulado
    \item Convertirlo a lenguaje matemático si es que no lo está
    \item Asumir lo contrario al postulado
    \item Usar álgebra o teoremas para demostrar que el postulado contrario es falso
    \item Queda probado que el postulado original es correcto
\end{enumerate}
\newpage
\subsection{Prueba por Inducción}
\textbf{Pasos:}
\begin{enumerate}
    \item Identificar el postulado
    \item Convertirlo a lenguaje matemático si es que no lo está y expresarlo en función de una variable ($n$)
    \item $P(n=1)$: Demostrar que se cumple el postulado para $n$=1
    \item $P(n=k)$: Asumir que se cumple el postulado para un natural arbitrario $n$=$k$ $\forall k\in\N$
    \item Paso inductivo $P(n=k+1)$:Usar el supuesto $P(n=k)$ para demostrar que el postulado se cumple también para $n=k+1$
    \item Queda probado que el postulado original es valido para cualquier $n\in\N$
\end{enumerate}
\end{document}